%% \chapter[htoc-titlei][hhead-titlei]{htitlei}
%% -----------------------------------------------------------------------------
\chapter[Alignment of the ATLAS Inner Detector][Alignment of the ATLAS Inner Detector]{Alignment of the ATLAS Inner Detector}
\label{ch:alignment}

In order for the subdetectors of the ID to operate at their designed precisions, it is essential that the locations of the sensors be known as precisely as possible.
Differences between the expected and actual positions of a sensor can result in displaced particle hits and degrade track reconstruction quality.
These misalignments can occur for any number of reasons, including but not limited to elemnts shifting during maintenance periods or cycles in ATLAS's magnetic field, or simply small movements during normal detector operations.
Since it is not practical to physically realign hundreds of thousands of detector elements to $\mu$m precision by hand, an iterative track-based alignment algorithm is used to determine the physical positions and orientations of these elements \cite{2011.alignment-7tev}. 
The effects of misalignments and the steps taken to correct and monitor them are detailed in this chapter.

\section{Effects of Misalignment}\label{align:effects}
Hello world!


\section{The Alignment Method}\label{align:method}
Hello world!


\section{Momentum Bias Corrections}\label{align:bias}
Hello world!


\section{Alignment of the IBL}\label{align:ibl} % does this need a dedicated section?
Hello world!


\section{Alignment Monitoring}\label{align:monitoring}
Hello world!

