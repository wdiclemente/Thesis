As no real HL-LHC data will be available for many years, all processes in this prospects study must be simulated using Monte Carlo (MC) generators. 
Signal and background processes were generated at \com{14}, and the event yields scaled to the anticipated HL-LHC integrated luminosity of $\mathcal{L}=3000~\textrm{fb}^{-1}$.

\TODO{Consider putting all this in a table}

%signal
The signal sample consists of both VBS and non-VBS electroweak (EWK) \ssww production, and it is sumulated with the \mcatnlo generator \cite{2014.madgraph_mcnlo} using the \nnpdf PDF set \cite{2015.NNPDF3} and interfaced with \pythiav{8} \cite{2015.pythia82} for hadronization and parton showering.
To study the longitudinal polarization more directly, two additional \mcatnlo \ssww samples are used: one containing only the longitudinal contribution (LL) and a second containing the transverse (TT) and mixed (LT) contributions.

%background
\TODO{Here we talk about things that mimic the experimental signature before we formally state what the signal is...}
There are many other processes that can produce the same final state as the \ssww and must also be accounted for using MC simulations.
$WZ$ events are generated using \sherpav{2.2.0} \cite{2009.Sherpa, 2008.CS_Shower, 2009.METS}, which includes up to one parton at next-to-leading order (NLO) in the strong coupling constant $\alpha_s$ and up to three additional partons at leading order (LO).  
Both EWK and QCD production are included in these samples.
$ZZ$ events are generated using \sherpav{2.2.2} with up to two additional partons in the final state.
Triboson backgrounds $VVV, V = W, Z$ where the bosons can decay leptonically or hadronically are simulated with \sherpav{2.2.2} with up to two additional partons in the final state.
$W$+jets backgrounds are generated for electron, muon, and tau final states are generated at LO with \mcatnlo and the \nnpdf set with showering from \pythiav{8}.
$Z$+jets events are generated using \powhegbox \cite{2010.powhegbox} and the \ctten PDF set \cite{2010.ct10} interfaced with \pythiav{8}.
Finally, $t\bar{t}$ and single-top events are generated using \powhegbox with showering from \pythiav{6}.

Since the MC samples used in the analysis are generated at particle-level and have not been run through the typical full simulation of the ATLAS detector, smearing functions are instead used to estimate detector effects.
These are derived from a \tt{GEANT4} simulation of the detector \cite{2003.GEANT4}.
%All MC samples are generated at particle-level, and detector effects are estimated using smearing functions derived from a full simulation of the ATLAS detector in \tt{GEANT4} \cite{2003.GEANT4}.
%Pileup events are fully simulated.
