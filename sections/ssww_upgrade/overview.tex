\subsection{Analysis Overview}
% define signal in workds
The experimental signature of interest here is identical to the $13\tev$ analysis detailed in Chapter~\ref{ch:ssww13tev}: two prompt leptons (electrons or muons) with the same charge, missing transverse energy, and two jets.
Once again the two leading jets are required to have a large angular separation and a high combined invariant mass to preferentially select EWK VBS production over QCD \ssww events.

% introduce key backgrounds
Background processes that can mimic the signal are again similar to the $13\tev$ analysis. %i am assuming prompt and non-prompt will be defined in the 13 TeV section
The dominant source of prompt background from $WZ$+jets events where both bosons decay leptonically.  
If the lepton from the $Z$-decay with opposite charge from the $W$ falls outside of the detector acceptance or is not identified, the remainder could appear to be a \ssww signal event.
To a lesser extent, $ZZ$+jets events can enter the signal region in much the same way provided two leptons are ``lost''.
Other prompt sources include $t\bar{t}+V$ and and multiple parton interactions, however these processes do not contribute much.
The upgrades to the ATLAS detector are expected to reduce the size of these prompt contributions due in large part to the increased detector acceptance from the forward tracking.
Jets mis-reconstructed as leptons or leptons from hacronic decays (such as $t\bar{t}$ and $W$+jets production) comprise the non-prompt lepton background.
Lastly, events with two prompt, opposite-charge electrons can contribute provided one of the electrons is mis-reconstructed as the wrong charge.

% goals of analysis -- xsec, polarization, optimization
In this analysis, the EWK production of \ssww is studied in the context of the planned HL-LHC run conditions and upgraded ATLAS detector.
An optimized event selection (referred to as the \emph{optimized selection}) is also explored in an effort to gain increased signal significance over the \emph{default selection}.
The cross section of the inclusive EWK production is measured for both the default and optimized selections, and the extraction of the longitudinal scattering significance is measured with the optimized selection.
