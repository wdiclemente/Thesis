\subsection{Analysis Overview}
% define signal in workds
The experimental signature of interest is identical to the $13\tev$ analysis (see Chapter~\ref{ch:ssww13tev}): two prompt leptons (either electrons or muons) with the same electric charge, missing transverse energy, and two high energy forward jets.
These jets are again required to have a large angular separation and a high combined invariant mass to preferentially select EWK-produced \ssww events.

% introduce key backgrounds
Background processes are not expected to change with respect to the $13\tev$ analysis and are summarized here. %i am assuming prompt and non-prompt will be defined in the 13 TeV section
The dominant source of prompt background comes from $WZ$+jets events where both bosons decay leptonically.  
If the lepton from the $Z$-decay with opposite charge from the $W$ falls outside of the detector acceptance or is not identified, the remaining two leptons will form a same-sign pair, and the event may pass the signal lepton criteria.
To a much lesser extent, $ZZ$+jets events can enter the signal region this way provided two leptons are ``lost''.
Other prompt sources include $t\bar{t}+V$ and and multiple parton interactions, however both contributions are small.
Overall, prompt backgrounds are expected to contribute less than in current analyses due to the addition of forward tracking in the upgraded ATLAS detector reducing the probability of leptons falling outside the detector acceptance.
Jets mis-reconstructed as leptons or leptons from hacronic decays (such as $t\bar{t}$ and $W$+jets production) comprise the non-prompt lepton background.
Lastly, events with two prompt, opposite-charge electrons can appear as a same-sign event provided one of the electrons is mis-reconstructed and assigned the wrong charge.

% goals of analysis -- xsec, polarization, optimization
In this analysis, the EWK production of \ssww is studied in the context of the planned HL-LHC run conditions and upgraded ATLAS detector.
An optimized event selection (referred to as the \emph{optimized selection}) is also explored in an effort to gain increased signal significance over the \emph{default selection}. %which is taken with minimal changes from the previous upgrade study~\cite{}.
The cross section of the inclusive EWK production is measured for both the default and optimized selections, and the extraction of the expected longitudinal scattering significance is measured with the optimized selection.
