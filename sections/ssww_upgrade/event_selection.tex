The majority of the default object and event selections were determined in the preceding $W^{\pm}W^{\pm}jj$ HL-LHC prospects study~\cite{2017.ssww-upgrade}, which focused on the impact of the upgraded detector's forward tracking capabilities.
Several different combinations of lepton and jet $\eta$ ranges were tested, and the results are used in this study. 

\subsection{Object selection}\label{sswwupgrade:object_selection}
Electrons and muons are preselected to have $\pt > 7$ and $6\gev$, respectively, and lie within $|\eta| \le 4.0$.
The likelihood of a given lepton to pass the trigger and identification requirements is estimated by calculating an efficiency dependent on the $\pt$ and $\eta$ of the lepton.
The leptons are also required to pass the isolation criteria detailed in Table~\ref{tab:truth_iso_definition}.
Jets that have been tagged as a fake electron by the functions described in Section~\ref{sswwupgrade:background} are treated as electrons for the purpose of the object selection and are subject to the same criteria.
In order to be considered a signal lepton, the transverse momentum requirement is raised to $\pt > 25\gev$.
The two highest $\pt$ leptons passing this selection are chosen to be the leading and subleading signal leptons.

Jets are clustered using the anti-$k_t$ algorithm from final-state particles (excluding muons and neutrinos) within a radius of $\deltar = 0.4$.
All jets are required to have $\pt > 30\gev$ and lie within $|\eta| < 4.5$; in order to suppress jets from pileup interactions, jets outside of $|\eta| \ge 3.8$ must pass an higher momentum cut of $\pt > 70\gev$.
Jets overlapping a preselected electron within $\deltar(e,j) < 0.05$ are removed in order to prevent double counting.
The two highest $\pt$ jets are defined as the leading and subleading tag jets.
% do i talk about jet vertex fraction?

%To study the expected impact of the upgraded detector's forward tracking, several different combinations of jet and lepton $\eta$ ranges were tested in \cite{2017.ssww-upgrade} and the results are used here.

\subsection{Event selection}\label{sswwupgrade:event_selection}

\begin{table}[htbp]
  \centering
  \begin{tabular}{l|c}
    Selection requirement              & Selection value \\
    \hline\hline
    \multirow{2}{*}{Lepton kinematics} & $\pt > 25\gev$ \\
                                       & $|\eta| \le 4.0$ \\
    \multirow{2}{*}{Jet kinematics}    & $\pt > 30\gev$ for $|\eta| \le 4.5$ \\
                                       & $\pt > 70\gev$ for $|\eta| > 3.8$ \\
    \hline
    Dilepton charge                    & Exactly two signal leptons with same charge\\
    Dilepton separation                & $\Delta R_{l,l} \ge 0.3$ \\
    Dilepton mass                      & $m_{ll} > 20\gev$\\
    $Z$ boson veto                     & $|m_{ee} - m_Z| > 10\gev$ ($ee$-channel only) \\
    $\met$                             & $\met > 40\gev$ \\
    Jet selection                      & At least two jets with $\Delta R_{l,j} > 0.3$\\
    $b$ jet veto                       & $N_{\textrm{b-jet}} = 0$\\
    Dijet separation                   & $\detajj > 2.5$\\
    Trilepton veto                     & No additional preselected leptons\\
    Dijet mass                         & $m_{jj} > 500\gev$\\
    Lepton-jet centrality              & $\zeta > 0$\\
    \hline
  \end{tabular}
  \caption{Summary of the signal event selection.}
  \label{tab:sswwupgrade_event_selection}
\end{table}

The default event selection is summarized in Table~\ref{tab:sswwupgrade_event_selection} and described here.
Exactly two signal leptons are required with the same electric charge and separated from each other by $\deltar(ll) > 0.3$.
In order to suppress contributions from Drell-Yan backgrounds, the two signal leptons must have an invariant mass $m_{ll}$ greater than $20\gev$.
Additionally, if both signal leptons are electrons, their mass must be at least $10\gev$ away from the $Z$-boson mass in order to reduce background from $Z$-boson decays\footnote{The electron charge misidentification rate in the upgraded ATLAS detector is estimated to be high enough that contributions from $Z\rightarrow ee$ backgrounds are non-negligible.}.
The event is required to have at least $40\gev$ of missing transverse energy ($\met$) to account for the two final-state neutrinos.
Events with additional preselected leptons are vetoed, which greatly reduces $WZ$ and $ZZ$ backgrounds.

Each event must have at least two jets, and both tag jets are required to not overlap with the signal leptons.
Events with one or more $b$-jets are vetoed to suppress backgrounds from heavy-flavor decays.
In order to preferentially select EWK production, the tag jets are required to have a large separation between them and a large invariant mass.
Finally, a cut on the lepton centrality $\zeta$\footnote{$\zeta$ is a measurement of whether the two signal leptons lie between the two tagging jets in $\eta$, as is preferred by the VBS topology.}, defined in Equation~\ref{eq:lepton_jet_centrality}, further enhances the EWK \ssww signal: %$\zeta$ is a measurement of whether the two signal leptons lie between the two tagging jets in $\eta$, as is preferred by the VBS topology.
\begin{equation}
\zeta = \min [\min (\eta_{\ell1}, \eta_{\ell2} )-\min(\eta_{j1},\eta_{j2}), \max(\eta_{j1},\eta_{j2})-\max(\eta_{\ell1},\eta_{\ell2}) ]\,.
\label{eq:lepton_jet_centrality}
\end{equation}
