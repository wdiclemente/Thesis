%Modern particle physics is generally interpreted in terms of the Standard Model (SM).
%This is a quantum field theory which encapsulates our understanding of the electromagnetic, weak, and strong interactions...

The Standard Model of particle physics serves as a mathematical description of the fundamental particles of the universe and their interactions.
It has been developed over the course of the past century, incorporating both predictions from theory and results from experiments.
All in all, the SM has proven remarkably successful in accurately describing particle interactions seen in experiments.

The SM is a quantum field theory (QFT)~\cite{1995.Weinberg, 1995.Peskin} in which the fundamental particles are represented as excited states of their corresponding fields.
The spin-$\frac{1}{2}$ fermionic fields give rise to the quarks and leptons comprising ordinary matter, the spin-1 fields correspond to the electroweak bosons and the gluon which mediate the electroweak and strong forces, respectively, and finally the scalar Higgs field is responsible for electroweak symmetry breaking.
The excitations and interactions of the fields are governed by the SM Lagrangian, which is invariant under local transformations of the group $SU(3)\times SU(2)\times U(1)$.
\TODO{need more detail/refinement here}

The first quantum field theory to be developed is quantum electrodynamics (QED)~\cite{1950.Feynman.QED}, which describes the electromagnetic interaction.
The theory predicts the existence of a $U(1)$ gauge field that interacts with the electrically charged fermions.
This field corresponds to the photon.
A key aspect of QED is that it is perturbative.
The coupling constant $\alpha = e^2/4\pi$ is small, where $e$ is electrical charge of the field, allowing for the use of perturbation theory in calculations.
In this case, calculations can be written as a power series in $\alpha$, where successive higher order terms contribute less to the final result.
\TODO{renormalizability here?}

Similar to QED, the strong interaction has also been described using QFT as quantum chromodynamics (QCD).
QCD is the theory of quarks and the associated strong force-carrier the gluon.
The symmetry group for QCD is $SU(3)$, and its eight generators correspond to the eight differently charged, massless gluons~\cite{1965.Han-Nambu.Propose-color}.
Unlike in QED, which has two charges, the strong force has three---called ``colors''.
Color charge combined with the non-Abelian nature of $SU(3)$, which allows the gluons to interact with each other, result in the most well-known property of QCD: color confinement.
In order to increase the separation between two color-charged particles, the amount of energy required increases until it becomes energetically favorable to pair-produce a new quark-antiquark pair, which then bind to the original quarks.
The end result of this is that only color-neutral objects exist in isolation. % and it is responsible for the hadronic jets in experiments.
A second important property of the QCD is that, unlike in QED, the coupling constant $\alpha_s$ is large at low energies where confinement occurs, and thus it cannot be described accurately using perturbation theory.
Fortunately, $\alpha_s$ ``runs'', or decreases at higher energy scales~\cite{1973.asymptotically-free-i, 1974.asymptotically-free-ii}; this allows QCD to be calculated perturbatively~\cite{1973.strong-perturbation} at energies accessible by collider experiments including the LHC.

The final force is the weak interaction.
It has an $SU(2)$ symmetry corresponding to three gauge bosons.
Ultimately, the weak and QED gauge fields mix and form the $SU(2)\times U(1)$ \emph{electroweak} interaction with four total bosons: three massive weak bosons ($W^\pm$ and $Z$) and the massless photon.


%\begin{itemize}
%  \item intro: theory+experimental motivation, four forces, blah
%  \item paragraph about what QFT is
%  \item Gauge invariance, renormalizibility of lagrangian (useful for the WW stuff later)
%  \item Electromagnetism - QED, $\alpha$, perturbation theory
%  \item Electroweak - mixing with EM, W/Z/$\gamma$, self-interacting is important for T/QGC, left handed
%  \item Strong - confinement, asymptotic freedom
%  \item Higgs - introduce it here w/ separate section after, source of mass for fermion/boson, discovery reference
%\end{itemize}
