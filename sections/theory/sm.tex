%Modern particle physics is generally interpreted in terms of the Standard Model (SM).
%This is a quantum field theory which encapsulates our understanding of the electromagnetic, weak, and strong interactions...

The Standard Model of particle physics serves as a mathematical description of the fundamental particles of the universe and their interactions.
It has been developed over the course of the past century, incorporating both predictions from theory and results from experiments.
All in all, the SM has proven remarkably successful in accurately describing particle interactions seen in experiments.

The SM is a quantum field theory (QFT)~\cite{1995.Weinberg, 1995.Peskin} in which the fundamental particles are represented as excited states of their corresponding fields.
The spin-$\frac{1}{2}$ fermionic fields give rise to the quarks and leptons comprising ordinary matter, the spin-1 fields correspond to the electroweak bosons and the gluon which mediate the electroweak and strong forces, respectively, and finally the scalar Higgs field is responsible for electroweak symmetry breaking.
The excitations and interactions of the fields are governed by the SM Lagrangian, which is invariant under local transformations of the group $\gtsu{3}\times\gtsu{2}\times\gtu{1}$.
\TODO{need more detail/refinement here}

The first quantum field theory to be developed is quantum electrodynamics (QED)~\cite{1950.Feynman.QED}, which describes the electromagnetic interaction.
The theory predicts the existence of a $\gtu{1}$ gauge field that interacts with the electrically charged fermions.
This field corresponds to the photon.
A key aspect of QED is that it is perturbative.
The coupling constant $\alpha = e^2/4\pi$ is small, where $e$ is electrical charge of the field, allowing for the use of perturbation theory in calculations.
In this case, calculations can be written as a power series in $\alpha$, where successive higher order terms contribute less to the final result.
\TODO{renormalizability here?}

The strong interaction---the theory of quarks and gluons---has also been described using QFT as quantum chromodynamics (QCD).
%QCD is the theory of quarks and the associated strong force-carrier the gluon.
The symmetry group for QCD is $\gtsu{3}$, and its eight generators correspond to the eight differently charged, massless gluons~\cite{1965.Han-Nambu.Propose-color}.
Unlike in QED, which has positive and negative charges, the strong force has three ``colors''.
Color charge combined with the non-Abelian nature of $SU(3)$, which allows the gluons to interact with each other, result in the most well-known property of QCD: color confinement.
In order to increase the separation between two color-charged quarks, the amount of energy required increases until it becomes energetically favorable to pair-produce a new quark-antiquark pair, which then bind to the original quarks.
The end result of this is that only color-neutral objects exist in isolation. % and it is responsible for the hadronic jets in experiments.
What this means for the strong coupling constant $\alpha_s$ is that its value at the low energies where confinement occurs is large, on the order of $\alpha_s\sim 1$.
The consequense of this is that perturbation theory cannot be used to accurately approximate interactions.
While this appears at first to be a critical problem for predictions, fortunately it turns out that $\alpha_s$ ``runs'', or decreases in magnitude at higher energy~\cite{1973.asymptotically-free-i, 1974.asymptotically-free-ii}.
This so-called ``asymptotic freedom'' allows QCD to be calcualted perturbatively~\cite{1973.strong-perturbation} at energies accessible by collider experiments including the LHC.
%A second important property of the QCD is that, unlike in QED, the coupling constant $\alpha_s$ is large at low energies where confinement occurs, and thus it cannot be described accurately using perturbation theory.
%Fortunately, $\alpha_s$ ``runs'', or decreases at higher energy scales~\cite{1973.asymptotically-free-i, 1974.asymptotically-free-ii}; this allows QCD to be calculated perturbatively~\cite{1973.strong-perturbation} at energies accessible by collider experiments including the LHC.

The last gauge field corresponds to the weak interaction.
Ultimately, the weak $\gtsu{2}$ and the electromagnetic $\gtu{1}$ mix to form the $\gtsu{2}\times\gtu{1}$ \emph{electroweak} (EWK) interaction~\cite{1959.Salam.electroweak, 1959.Glashow.vector-mesons}.
A more detailed description of the mixing will be discussed in conjunction with electroweak symmetry breaking (EWSB) in Section~\ref{sec:theory_higgs}; however, a summary of the resulting EWK interaction is presented here, at the risk of some repeated information to follow.
There are three weak isospin bosons arising from the $\gtsu{2}$ group ($W_{\mu}^1$, $W_{\mu}^2$, and $W_{\mu}^3$) and one weak hypercharge boson from the $\gtu{1}$ group ($B_{\mu}$).
The $W_3$ and $B$ bosons mix according to the weak mixing angle $\theta_W$ to form the $Z$ boson and the photon according to:
\begin{equation}
  \begin{pmatrix}
  \gamma\\
  Z
  \end{pmatrix}
  = 
  \begin{pmatrix}
  \cos{\theta_W} & \sin{\theta_W} \\
  -\sin{\theta_W} & \cos{\theta_W}
  \end{pmatrix}
  \begin{pmatrix}
  B_{\mu} \\
  W_{\mu}^3
  \end{pmatrix}
  \label{eq:weak_mixing}
\end{equation}
The value of $\theta_W$ is not predicted by the SM; it is one example of an experimental input to the theory, measured to be $\sin^2{\theta_W} = 0.23153\pm0.00016$~\cite{2006.weak-mixing}.
The charged $W^{\pm}$ bosons are a mixture of the remaining $W_{\mu}^1$ and $W_{\mu}^2$ bosons:
\begin{equation}
  W^{\pm} = \frac{1}{\sqrt{2}}(W_{\mu}^1 \mp iW_{\mu}^2)
  \label{eq:w_mixing}
\end{equation}
Unlike the photon (and the gluon of QCD), the $W^\pm$ and $Z$ bosons are massive.
This means that even though $\gtsu{2}$ is non-Abelian, the range of interaction is short and confinement does not occur.
Lastly, the EWK interaction is chiral, only coupling to the left-handed component of the fermion fields

One final field remains within the SM: the scalar Higgs field.
It was originally proposed in the 1960's to explain the masses of the $W^{\pm}$ and $Z$ bosons~\cite{1964.Englert.symmetry_breaking, 1964.Higgs.Broken_Symmetries_1, 1964.Higgs.Broken_Symmetries_2} and is the mechanism for the EWSB process.
The particle associated with the field is a massive scalar boson, which was at last discovered by ATLAS and CMS in 2012~\cite{HIGG-2012-27, CMS-HIG-12-028} with a mass of $125\gev$.


%\subsection{Electroweak mixing}\label{theory:ewk_mixing}
%\begin{itemize}
%  \item intro: theory+experimental motivation, four forces, blah
%  \item paragraph about what QFT is
%  \item Gauge invariance, renormalizibility of lagrangian (useful for the WW stuff later)
%  \item Electromagnetism - QED, $\alpha$, perturbation theory
%  \item Electroweak - mixing with EM, W/Z/$\gamma$, self-interacting is important for T/QGC, left handed
%  \item Strong - confinement, asymptotic freedom
%  \item Higgs - introduce it here w/ separate section after, source of mass for fermion/boson, discovery reference
%\end{itemize}
