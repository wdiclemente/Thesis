%% \chapter[htoc-titlei][hhead-titlei]{htitlei}
%% -----------------------------------------------------------------------------
\chapter[Introduction][Introduction]{Introduction}
% If you want space after a footnote, you need to force it like so.
%The Standard Model (SM)\footnotemark~has been remarkably successful...
%\footnotetext{Here's a footnote.}
The Large Hadron Collider (LHC) at CERN is the most powerful collider experiment in the world.
At the time of its construction, the largest unanswered question in the Standard Model (SM) was the mechanism behind electroweak symmetry breaking (EWSB).
As a result, one of the primary goals of the experiment is to learn as much as possible about this mechanism.
Thus far, the LHC has suceeded in discovering a particle consistent with the long-awaited Higgs boson. %one of the largest missing pieces in the Standard Model (SM) of particle physics.
In addition, measurements of many SM processes have been performed for the first time or at better precision than before thanks to the high collision energy and large volume of data collected by the LHC.

Processes involving the scattering of two massive electroweak (EWK) gauge bosons are of particular interest at the LHC for two main reasons.
Firstly, they allow for tests of the self-interactions predicted by the EWK gauge theory through triple and quartic gauge couplings.
While the triple couplings have been studied by previous experiments as well as at the LHC, the quartic couplings of the massive gauge bosons have not been accessible previously.
Thus, processes involving these couplings can be measured and compared to the SM predictions for the first time.
Secondly, the scattering of two massive gauge bosons is sensitive to the underlying EWSB mechanism.
The $W^{\pm}$ and $Z$ bosons are given non-zero masses---and consequently a longitudinal polarization mode---through the Higgs mechanism, and thus their interactions serve as a direct probe of the symmetry breaking sector.

This thesis presents two separate analyses dealing with the scattering of two same-sign $W^{\pm}$ bosons with the LHC's ATLAS experiment.
The \ssww process is one of the most sensitive to the goals above: it has access to the $WWWW$ quartic gauge coupling, production modes that involve the exchange of a Higgs boson, and relatively low backgrounds.
Evidence of EWK \ssww production was first seen by the ATLAS and CMS experiments at \com{8}, however the data set was too small to claim observation of the process.
The first analysis covered here is the follow up to the above ATLAS measurement, measuring the EWK fiducial cross section at \com{13} with a larger data sample.
The second analysis explores the prospects for future measurements of the \ssww process at the planned High-Luminosity LHC (HL-LHC).
A measurement of the production cross section as well as sensitivity to the purely longitudinal component of the $W^{\pm}W^{\pm}$ scattering is presented.

In addition to the SM measurements, a part of this thesis is devoted to the detector components making up ATLAS's Inner Detector (ID).
Precise knowledge of the locations of detector elements is essential for accurate particle track reconstruction, which in turn results in improved resolutions for physics measurements.
Special emphasis is given to the monitoring of momentum biases that may exist in the ID even after alignment.

This first few chapters of this thesis are intended to provide context for the main topics.
Chapter~\ref{ch:theory} gives a brief introduction to the Standard Model with a focus on the mechanism of electroweak symmetry breaking and vector boson scattering.
The experimental apparatus---the LHC and the ATLAS detector---are detailed in Chapter~\ref{ch:detector}.
The next three chapters present the main body of work.
Chapter~\ref{ch:alignment} covers the alignment of the ATLAS Inner Detector.
Finally, Chapters~\ref{ch:ssww13tev} and \ref{ch:sswwupgrade} detail the ATLAS \com{13} \ssww cross section measurement and the \com{14} HL-LHC \ssww prospects study, respectively.

%\TODO{Make sure to mention previous ATLAS measurements of QGC's (wgg and 8 tev ssww) in brief discussion of why these diboson processes are interesting}
