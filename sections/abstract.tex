This thesis presents two studies of electroweak same-sign \ssww scattering performed with the ATLAS experiment at the CERN Large Hadron Collider (LHC).
The first is a measurement of the fiducial cross section at \com{13} using $36.1~\mathrm{fb}^{-1}$ of data recorded in 2015 and 2016.
Particular emphasis is placed on the determination of the non-prompt lepton background.
The electroweak production is observed with a signal significance of $6.9\sigma$, and the fiducial cross section is measured to be $\sigma_{\textrm{meas}}^{\textrm{fid}}~=~2.91^{+0.51}_{-0.47}(\textrm{stat})~^{+0.28}_{-0.29}(\textrm{sys})~\textrm{fb}$.
The second is a study on the future prospects for the \ssww process at the planned High-Luminosity LHC with a projected $\sqrt{s}=14\tev$ and $3000~\mathrm{fb}^{-1}$ of data, with a focus on the optimization of the signal event selection.
The expected electroweak production cross section is determined with a total uncertainty  of $6\%$, and the purely longitudinal scattering component is extracted with an expected significance of $1.8\sigma$.
This thesis also describes work performed on the alignment of the ATLAS Inner Detector, which is used to reconstruct the trajectories of charged particles and to determine their momenta.
A precise alignment is essential for the majority of physics results from the ATLAS experiment.
The main topics presented are the alignment campaign performed at the beginning of the 2015 data taking period and the monitoring of momentum biases in the Inner Detector.
%Additionally, some time is taken to detail the alignment of the ATLAS Inner Detector subsystems, as good alignment performance is essential for making high-quality physics measurements.
