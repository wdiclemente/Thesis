%% \chapter[htoc-titlei][hhead-titlei]{htitlei}
%% -----------------------------------------------------------------------------
\chapter[Conclusion][Conclusion]{Conclusion}

This thesis presented a measurement of the fiducial cross section of EWK production of same-sign $W$ boson pairs at \com{13} with the ATLAS experiment as well as prospects for the measurement at the planned High-Luminosity LHC.
The \ssww EWK process is observed in $36.1~\textrm{fb}^{-1}$ of proton-proton collision data recorded by ATLAS at $13\tev$ with a signal significance of $6.9\sigma$.
The fiducial cross section is measured to be $\sigma_{\textrm{meas}}^{\textrm{fid}} = 2.91^{+0.51}_{-0.47}(\textrm{stat})~^{+0.12}_{-0.16}(\textrm{theory})~^{+0.24}_{-0.23}(\textrm{sys})~^{+0.08}_{-0.06}(\textrm{lumi})~\textrm{fb}$.
The future prospects of the measurement at the HL-LHC at $14\tev$ and a planned $3000~\textrm{fb}^{-1}$ of data is predicted to have a precision of $6\%$.
In addition, the significance of a potential measurement of the purely longitudinally polarized component of \sswwnojj scattering is predicted to be $1.8\sigma$.

In addition, the track-based alignment of the ATLAS Inner Detector is described in detail as well as its applications during Run 2.
The alignment campaign using the first \com{13} $pp$ collision data in early 2015 was described in detail, and it provided the baseline alignment used for the remainder of the year.
The monitoring of momentum biases using electron $\eop$ was aslo presented, including results on sagitta biases in the Inner Detector after the reprocessing of the 2016 data.

%\paragraph*{Looking Ahead} \hspace{0pt} \\
%
%Here's an example of how to have an ``informal subsection''.
