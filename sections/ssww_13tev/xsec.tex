The \ssww EWK cross section is extracted from the signal region using a maximum-likelihod fit applied simultaneously to four $m_{jj}$ bins in the signal region as well as to the low-$m_{jj}$ and $WZ$ control regions.
For the cross section extraction, the signal region is defined as in Table~\ref{tab:ssww13tev_event_selection} with the dijet invariant mass requirement raised to $m_{jj} > 500\gev$.
The low-$m_{jj}$ region is defined to mirror the signal region exactly with the dijet invariant mass inverted to $200 < m_{jj} < 500\gev$, and the $WZ$ control region is defined previously in Section~\ref{ssww13tev:wz}.

The signal and low-$m_{jj}$ regions are split into six channels based on the flavor and charge of the dilepton pair: $\mmp$, $\mmm$, $\mep$, $\mem$, $\eep$, and $\eem$.
This split by charge increases the sensitivity of the measurement due to the $W^{+}/W^{-}$ charge asymmetry at hadron colliders favoring the production of $W^{+}$ bosons~\cite{2010.w-charge-asymmetry}.
Since the signal events contain two $W$ bosons, the signal strength compared to charge-symmetric backgrounds is much greater in the $++$ channels for both charges combined.
The $WZ$ control region is included in the fit as a single bin ($l^{\pm}l^{\pm}l^{\pm}$).

The maximum likelihood fit is outlined in Section~\ref{ssww13tev:xsec_fit_method} and the cross section extraction method in Section~\ref{ssww13tev:xsec_method}.

\subsection{Maximum likelihood fit}\label{ssww13tev:xsec_fit_method}
\TODO{This section is very similar to what is written in the support note... May need to put some work into flushing it out so it's not so close to copy-paste}
The number of predicted signal events in each channel $c$ and $m_{jj}$ bin $b$ can be calculated from the SM predicted fiducial signal cross section $\sigma^{\textrm{fid}}$, the total integrated luminosity $\mathcal{L}$, the signal acceptance $\mathcal{A}_b$, and the efficiency corrections $\mathcal{C}(\theta)$, where $\theta$ represents the set of nuisance parameters that parameterize the effects of each systematic uncertainty on the signal and background expectations.
\begin{equation}
  N_{cb}^{\textrm{sig}}(\theta) = \sigma^{\textrm{fid}}\mathcal{A}_b\mathcal{C}_{b}(\theta)\mathcal{L}
  \label{eq:ssww13tev_xsec_nsig}
\end{equation}
A signal strength parameter $\mu$ is defined as the ratio of the measured cross section to the SM predicted cross section.
%it seems like $\mu$ is the main thing we're extracting from the fit, and we calculate the cross section by measuring $\mu$ and solving for $\sigma_obs$, maybe put some words about this in
The expected number of events in a given channel and bin can then be expressed as the sum of the estimated background ($N_{cb}^{\textrm{bkg}}(\theta)$) and the number of predicted signal events scaled by $\mu$:
\begin{equation}
  \begin{aligned}
    N_{cb}^{\textrm{exp}}(\theta) &= \mu N_{cb}^{\textrm{sig}}(\theta) + N_{cb}^{\textrm{bkg}}(\theta) \\
                              &= \mu \sigma^{\textrm{fid}}\mathcal{A}_b\mathcal{C}_{b}(\theta)\mathcal{L} + N_{cb}^{\textrm{bkg}}(\theta)
  \end{aligned}
  \label{eq:ssww13tev_xsec_nexp}
\end{equation}

The nuisance parameters are constrained by Gaussian probability distribution functions, and the normalization of the $WZ$ background mentioned in Section~\ref{ssww13tev:wz} is included in the fit as a free parameter.
The expected yields for signal and background processes are adjusted by the set of nuisance parameters within the constraints of the systematic uncertainties.
The yields after the fit correspond to the value that best matches the observed data.

The expected number of events per channel and bin can be written as a sum of the predicted event yields for each sample $s$:
\begin{equation}
  \nu_{cb}(\phi,\theta,\gamma_{cb}) = \gamma_{cb}\sum\limits_{s}\big[\eta_{cs}(\theta)\phi_{cs}(\theta)\lambda\big] h_{cbs}(\theta)
  \label{eq:ssww13tev_xsec_nexp_fit}
\end{equation}
In this equation, the expected number of events in a given channel and bin is obtained by weighting the histogram of predicted yields $h_{cbs}$ by the product of a given luminosity $\lambda$ and any normalization factors $\phi_{cs}$ that may be given for each channel and sample.
The input histogram and the normalization factors may depend on the nuisance parameters $\theta$ taking into account sources of systematic uncertainty.
Uncertainties on the normalization factors $\eta_{cs}(\theta)$ are also included.
Finally, bin-by-bin scale factors $\gamma_{cb}$ are included to parameterize the statistical uncertainties of the MC predictions.

The binned likelihood function is given by a product of Gaussian functions for the luminosity and for the background uncertainties and a product of Poisson functions for the number of observed events in each bin and channel:
\begin{equation}
  L(\mu|\theta) = \mathcal{G}(\mathcal{L}|\theta_{\mathcal{L}},\sigma_{\mathcal{L}})\cdot \prod\limits_{c}\prod\limits_{b}\mathcal{P}(N_{\textrm{cb}}^{\textrm{meas.}}|\nu_{cb}(\mu))\prod\limits_{p}\mathcal{G}(\theta_{p}^{0}|\theta_{p})
\end{equation}
where $\mathcal{G}$ and $\mathcal{P}$ are the Gaussian and Poisson functions, respectively.
As before, $\mathcal{L}$ represents the integrated luminosity with uncertainty $\sigma_{\mathcal{L}}$ and associated nuisance parameter $\theta_{\mathcal{L}}$.
The number of measured events in a given bin and channel is represented by $N_{cb}^{\textrm{meas.}}$, and $\nu_{cb}(\mu)$ is the predicted number of events defined in Equation~\ref{eq:ssww13tev_xsec_nexp_fit} expressed as a function of the signal strength $\mu$.
Finally, the set of nuisance parameters $\theta$ and any auxiliary measurements used to constrain them $\theta^0$ are multiplied for each parameter $p$.

The profile likelihood ratio is defined as
\begin{equation}
  q_{\mu} = -2\ln\frac{L(\mu,\hat{\hat{\theta}}_{\mu})}{L(\hat{\mu},\hat{\theta})}
  \label{dq:ssww13tev_xsec_test_statistic}
\end{equation}
with $\hat{\mu}$ and $\hat{\theta}$ as the unconditional maximum likelihood estimates and $\hat{\hat{\theta}}$ as the conditional maximum likelihood estimate for a given value of $\mu$.
%basically the idea here is that the unconditional maximum is just the best value of \theta and \mu.  the conditional maximum is the best value of \theta GIVEN a particular \mu (i.e. the best value of \theta is conditional upon the choice of \mu)
%Since $q_{\mu}$ is asymptotically $\chi^2$ distributed according to Wilks' theorem~\cite{}
The fitted signal strength $\hat{\mu}$ is obtained by maximizing the likelihood function with respect to all parameters.
The compatibility of the observed data with the background-only hypothesis can then be calculated by setting $\mu=0$.
Observation of the \ssww EWK process is claimed if the data is found to be inconsistent with the background-only hypothesis by more than $5\sigma$.
The observed signal strength $\mu_{\textrm{obs}}$ is obtained by using the real data yields in each input bin to the fit; calculating the expected signal strength $\mu_{\textrm{exp}}$ follows the same method but with the data replaced by the predictions from simulation and data-driven estimations.

\subsection{Cross section extraction}\label{ssww13tev:xsec_method}
\TODO{support note 9.9 p 164}

\subsubsection{Fiducial region definition}\label{ssww13tev:fiducial_def}

\begin{table}[htbp]
  \centering
  \begin{tabular}{c}
    
  \end{tabular}
  \caption{Definition of the fiducial volume.}
  \label{tab:ssww13tev_fiducial_vol}
\end{table}
