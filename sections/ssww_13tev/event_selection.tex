This section details the selection criteria for objects used in the analysis as well as the selection for signal events.

\subsection{Object selection}\label{ssww13tev:object_selection}
Muons, electrons, and jets all must pass strict selection requirements to ensure that only high quality, well measured objects are used.
For leptons, a baseline selection is defined (called the \emph{preselection}), which all leptons must pass in order to be considered for the analysis.
This preselection is an intentionally loose set of criteria in order to have high acceptance for rejecting backgrounds with additional leptons (i.e. $WZ\rightarrow 3l\nu jj$).
Signal leptons are then required to satisfy a much tighter \emph{signal selection} aimed at suppressing backgrounds from non-prompt or fake leptons.
A third set of lepton selection criteria, the \emph{loose selection}, defines a sample enriched in non-prompt leptons, and it is used in the fake factor method for estimating the non-prompt background, discussed in detail in Section~\ref{ssww13tev:fake_factor}.
Jets are only required to pass one set of selection criteria.
These selections are detailed in the following sections and summarized in Table~\ref{tab:ssww13tev_muon_selection} for muons, Table~\ref{tab:ssww13tev_elec_selection} for electrons, and Table~\ref{tab:ssww13tev_jet_selection} for jets.

\subsubsection{Muon candidate selection}
Cuts on muon $\pt$ serve to reject low momentum leptons from background processes and additional collisions from pileup events.
Preselected muons must have transverse momentum $\pt > 6\gev$, and the signal muons must pass $\pt > 27\gev$.
The $\pt$ requirement for loose muons is lower than for signal muons, $\pt > 15\gev$, for reasons that are discussed in Section~\ref{ssww13tev:fake_factor}.
Muons are required to fall within the detector's $\eta$ acceptance: $|\eta| < 2.7$ for preselected muons, which is tightened to $|\eta| < 2.5$ for the signal muons.

Cuts on the transverse and longitudinal impact parameters are applied to ensure that the candidate muon originated from the primary particle interaction and not some other source, such as a heavy flavor decay.
The preselection and the loose selection both have looser requirements on the transverse impact parameter significance ($d_0/\sigma_{d_{0}}$) than the signal selection; all three have the same requirement on the transverse impact paramter ($|z_0\times\sin\theta|$).

Finally, the muon candidates are required to pass a particle identification and an isolation criteria as defined in~\cite{2016.muon-reconstruction-13tev}.
The methods used in constructing the identification and isolation workingpoints are described in more detail in Section~\ref{detector:muon_reconstruction}.
The muon identification serves to select prompt muons with high efficiency and well measured momenta. % via the techniques detailed in Section~\ref{sec:reconstruction}.
This analysis uses two different workingpoints, \tt{Loose} for preselected muons and \tt{Medium} for loose and signal muons, where \tt{Medium} muons are a tighter subset of those that pass the \tt{Loose} requirement.
Muon isolation is a measurement of detector activity around the muon candidate, and it is measured with both track-based and calorimeter-based variables.
The isolation workingpoint used for the signal muons, \tt{Gradient}, is defined such that there is 90\% or better background rejection efficiency for $25\gev$ muons, and 99\% efficiency at $60\gev$.
There is no minimum isolation requirement for preselected or loose muons.
Loose muons are additionally required to fail one or both of the signal transverse impact parameter cut and signal isolation requirement.

\begin{table}[htbp]
  \centering
  \begin{tabular}{l l}
    \multicolumn{2}{c}{Muon preselection} \\ 
    \hline\hline
    Momentum cut                  & $\pt > 6~\textrm{GeV}$ \\
    Angular acceptance            & $|\eta| < 2.7$ \\
    Longitudinal impact parameter & $|z_0\times\sin\theta| < 0.5~\textrm{mm}$ \\
    Transverse impact parameter   & $d_0/\sigma_{d_{0}} < 10$ \\
    Particle identification       & \tt{Loose} \\
    \hline
  \end{tabular}

  \vspace{8mm}

  \begin{tabular}{l l}
    \multicolumn{2}{c}{Muon signal selection} \\ 
    \hline\hline
    Momentum cut                  & $\pt > 27~\textrm{GeV}$ \\
    Angular acceptance            & $|\eta| < 2.5$ \\
    Longitudinal impact parameter & $|z_0\times\sin\theta| < 0.5~\textrm{mm}$ \\
    Transverse impact parameter   & $d_0/\sigma_{d_{0}} < 3$ \\
    Particle identification       & \tt{Medium} \\
    Particle isolation            & \tt{Gradient}\\
    \hline
  \end{tabular}

  \vspace{8mm}

  \begin{tabular}{l l}
    \multicolumn{2}{c}{Muon loose selection} \\ 
    \hline\hline
    Momentum cut                  & $\pt > 27~\textrm{GeV}$ \\
    Angular acceptance            & $|\eta| < 2.5$ \\
    Longitudinal impact parameter & $|z_0\times\sin\theta| < 0.5~\textrm{mm}$ \\
    Transverse impact parameter   & $d_0/\sigma_{d_{0}} < 10$ \\
    Particle identification       & \tt{Medium} \\
    \multicolumn{2}{c}{Fail signal transverse impact parameter and/or isolation cuts} \\
    \hline
  \end{tabular}
  \caption{Muon selection criteria.  All muons are required to pass the preselection (top), and then either the signal (middle) or loose (bottom) criteria is applied to the preselected electrons.}
  \label{tab:ssww13tev_muon_selection}
\end{table}


\subsubsection{Electron candidate selection}
The electron candidate selections are very similar to those for muons.
The $\pt$ cut starts at $\pt > 6\gev$ for the preselection, increases to $\pt > 20\gev$ for loose electrons, and finally to $\pt > 27\gev$ for signal electrons.
The $|\eta|$ cut for electrons requires $|\eta| < 2.47$ for all electrons, with the region $1.37 \le |\eta| \le 1.52$ removed from loose and signal electrons.
This region is where the electromagnetic calorimeter transitions from the barrel to the endcaps and is not fully instrumented.
Both the transverse and longitudinal impact parameter cuts are the same for all electron selections.

The electron particle identification uses a multivariate likelihood technique (LH)~\cite{2016.electron-performance-13tev} detailed in Section~\ref{detector:electron_reconstruction}.
Preselected electrons must pass the loosest LH workingpoint \tt{LooseLH} with an additional requirement that there be a reconstructed track hit in the first layer of the pixel detector (a so-called $B$-layer hit).
The LH requirement for the loose and signal electrons the tighness of the identification using \tt{MediumLH} and \tt{TightLH}, respectively.
As for isolation, the \tt{Gradient} workingpoint is required for signal electrons only.
The loose electrons must fail one or both of the signal identification and isolation requirements.

\begin{table}[htbp]
  \centering
  \begin{tabular}{l l}
    \multicolumn{2}{c}{Electron preselection} \\ 
    \hline\hline
    Momentum cut                  & $\pt > 6\gev$ \\
    Angular acceptance            & $|\eta| < 2.47$ \\
    Longitudinal impact parameter & $|z_0\times\sin\theta| < 0.5~\textrm{mm}$ \\
    Transverse impact parameter   & $d_0/\sigma_{d_{0}} < 5$ \\
    Particle identification       & \tt{LooseLH} + $B$-layer hit \\
    \hline
  \end{tabular}

  \vspace{8mm}

  \begin{tabular}{l l}
    \multicolumn{2}{c}{Electron signal selection} \\ 
    \hline\hline
    Momentum cut                  & $\pt > 27\gev$ \\
    Angular acceptance            & $|\eta| < 2.47$, excluding $1.37 \le |\eta| \le 1.52$ \\
    Longitudinal impact parameter & $|z_0\times\sin\theta| < 0.5~\textrm{mm}$ \\
    Transverse impact parameter   & $d_0/\sigma_{d_{0}} < 5$ \\
    Particle identification       & \tt{TightLH} \\
    Particle isolation            & \tt{Gradient}\\
    \hline
  \end{tabular}

  \vspace{8mm}

  \begin{tabular}{l l}
    \multicolumn{2}{c}{Electron loose selection} \\ 
    \hline\hline
    Momentum cut                  & $\pt > 20\gev$ \\
    Angular acceptance            & $|\eta| < 2.47$, excluding $1.37 \le |\eta| \le 1.52$ \\
    Longitudinal impact parameter & $|z_0\times\sin\theta| < 0.5~\textrm{mm}$ \\
    Transverse impact parameter   & $d_0/\sigma_{d_{0}} < 5$ \\
    Particle identification       & \tt{MediumLH} \\
    \multicolumn{2}{c}{Fail signal identification and/or isolation cuts} \\
    \hline
  \end{tabular}
  \caption{Electron selection criteria.  All electrons are required to pass the preselection (top), and then either the signal (middle) or loose (bottom) criteria is applied to the preselected electrons.}
  \label{tab:ssww13tev_elec_selection}
\end{table}


\subsubsection{Jet candidate selection}
The final objects that need to pass selection are jets.
Jets are clustered using the anti-$k_t$ algorithm~\cite{2008.antikt} within a radius of $\deltar = 0.4$.
The jets are then calibrated using $E_\textrm{T}$- and $\eta$-dependent correction factors that are trained using MC simulations~\cite{2017.jet-energy-scale-13tev}.
These calibrated jets are then required to have $\pt > 30\gev$ if they lie in the forward regions of the detector ($2.4 < |\eta| < 4.5$) and $\pt > 25\gev$ in the central region ($|\eta| \le 2.4$).
In order to suppress pileup jets, the so-called jet-vertex-tagger (JVT) discriminant associates a jet with the primary interaction vertex~\cite{2014.jet-vertex-tagger}; central jets with $\pt > 60\gev$ are required to pass the \tt{Medium} JVT workingpoint, which corresponds to an average efficiency of over 92\%. %the exact value is JVT > 0.59 per https://twiki.cern.ch/twiki/bin/view/AtlasProtected/JVTCalibration#Working_points
Finally, the jets are required to be separated by selected prompt leptons by at least $\deltar(j,l) > 0.3$.

\begin{table}[htbp]
  \centering
  \begin{tabular}{l l}
    \multicolumn{2}{c}{Jet selection} \\ 
    \hline\hline
    \multirow{2}{*}{Momentum cut} & $\pt > 30\gev$ for $2.4 < |\eta| < 4.5$ \\
                                  & $\pt > 60\gev$ for $|\eta| < 2.4$ \\
    JVT cut                       & \tt{Medium}\\
    Jet-lepton separation         & $\deltar(j,l) > 0.3$ \\
    \hline
  \end{tabular}
  \caption{Jet selection criteria.  All jets are required to pass the above selection in order to be used in the analysis.}
  \label{tab:ssww13tev_jet_selection}
\end{table}

\subsubsection{Treatment of overlapping objects}\label{ssww13tev:overlap_removal}
In the event that one or more objects are reconstructed very close to each other, there is the possiblity for double-counting if both originated from the same object.
The procedure by which this ambiguity is resolved is called \emph{overlap removal} (OR).
The standard ATLAS recommendation for OR is implemented in this analysis~\cite{2014.atlas-overlap-removal, 2018.atlas-wboson-top} and is summarized in Table~\ref{tab:ssww13tev_or}.

Since electrons leave a shower in the EM calorimeter, every electron has a jet associated with it.
Therefore, any jets close to an electron (within $\deltar(e,j) < 0.2$) are rejected due to the high probability that they are the same object.
On the other hand, when jets and electrons overlap within a large radius of $0.2 < \deltar(e,j) < 0.4$, it is likely that the electron and jet both are part of a heavy-flavor decay, and the electron is rejected.

High energy muons can produce photons via bremsstrahlung radiation or collinear final state radiation which results in a nearby energy deposit in the calorimeters.
Non-prompt muons from hadronic decays produce a similar signature; however, in this case the jet has a higher track multiplicity in the ID.
It is possible to address both cases by rejecting the jet when the ID track multiplicity is less than three and otherwise rejecting the muon for jets and muons within $\deltar(\mu,j) < 0.4$.

In addition to the case above where muon bremsstrahlung results in a nearby reconstructed jet, the ID track from the muon and the calorimeter energy deposit can lead to it bein reconstructed as an electron.
In this case, if both a muon and an electron share a track in the ID, the muon is kept and the electron is rejected, unless the muon is calorimeter-tagged\footnote{A calorimeter-tagged (CT) muon is a muon that is identified by matching an ID track to a calorimeter energy deposit.  CT muons have relatively low reconstruction efficiency compared to those measured by the MS, but can be used to recover acceptance in regions of the detector where the MS does not have full coverage~\cite{2016.muon-reconstruction-13tev}.}, in which case the muon is removed in favor of the electron.

\begin{table}[htbp]
  \centering
  \begin{tabular}{l l l}
    Overlap         & Check & Result (remove $\rightarrow$ keep) \\
    \hline\hline
    \multirow{2}{*}{Electron \& Jet}  & $\deltar(e,j) < 0.2$       & Jet $\rightarrow$ Electron \\
                                      & $0.2 < \deltar(e,j) < 0.4$ & Electron $\rightarrow$ Jet \\
    \hline
    \multirow{2}{*}{Muon \& Jet}      & $\deltar(\mu,j) < 0.4$ and Jet $N_{\textrm{ID\ tracks}} < 3$
                                                                   & Jet $\rightarrow$ Muon \\
                                      & $\deltar(\mu,j) < 0.4$ and Jet $N_{\textrm{ID\ tracks}} \ge 3$
                                                                   & Muon $\rightarrow$ Jet \\
    \hline
    \multirow{2}{*}{Electron \& Muon} & Shared ID track            & Electron $\rightarrow$ Muon \\
                                      & Shared ID track \& muon is calo-tagged
                                                                   & Muon $\rightarrow$ Electron \\
    \hline
  \end{tabular}
  \caption{Summary of the overlap removal procedure used in the analysis.  If the criteria in the ``check'' column is met, in the ``result'' column, the object on the left of the arrow is removed in favor of the object on the right.}
  \label{tab:ssww13tev_or}
\end{table}

\subsection{Signal event selection}\label{ssww13tev:event_selection}
After the objects have been selected, cuts are applied on a per-event level to select \ssww signal events.
The event selection is summarized in Table~\ref{tab:ssww13tev_event_selection} and is detailed in this section.
It includes the results of an optimization performed using a multidimensional grid scan.

The initial event selection begins by choosing events that pass one or more of the trigger requirements listed in Table~\ref{tab:ssww13tev_triggers}.
At least one signal lepton is ``matched'' to a passed trigger in order to ensure that it was indeed a signal lepton that fired the trigger.
A collection of \emph{event cleaning} cuts must also be passed in order to remove events collected during periods in which one or more components of the detector was not operating optimally.
Finally, the events are required to contain at least one interaction vertex.
An event can have multiple reconstructed vertices from additional proton-proton collisions that occurred in the same bunch crossing.
In this case, the \emph{primary vertex} is determined by choosing the vertex with the largest sum of the $\pt^2$ of its associated tracks.

\begin{table}[htbp]
  \centering
  \begin{tabular}{l | c | c}
    & 2015 data & 2016 data \\
    \hline\hline
    \multirow{3}{*}{Electrons} & $\pt > 24\gev$ and \tt{Medium} ID & $\pt > 26\gev$ and \tt{Tight} ID and \tt{Loose} isolation\\
                               & $\pt > 60\gev$ and \tt{Medium} ID & $\pt > 60\gev$ and \tt{Medium} ID \\
                               & $\pt > 120\gev$ and \tt{Loose} ID & $\pt > 140\gev$ and \tt{Loose} ID \\
    \hline
    \multirow{2}{*}{Muons}     & $\pt > 20\gev$ and \tt{Loose} isolation & $\pt > 26\gev$ and \tt{Medium} isolation\\
                               & $\pt > 50\gev$ & $\pt > 50\gev$ \\
    \hline
  \end{tabular}
  \caption{Summary of trigger requirements for electrons and muons for \com{13} data collected in 2015 and 2016.  At least one of the triggers must be satisfied.}
  \label{tab:ssww13tev_triggers}
\end{table}

Events are then required to contain exactly two signal leptons with the same electric charge.
The dilepton pair must have a combined invariant mass of $m_{ll} \ge 20\gev$ in order to suppress low mass Drell-Yan backgrounds.
Two additional selections are applied to events in the $ee$-channel: both electrons are required to have $|\eta| < 1.37$ with an invariant mass at least $15\gev$ away from the $Z$-boson mass to reduce events where one electron is reconstructed with the wrong charge (this background will be discussed in more detail in Section~\ref{ssww13tev:charge_misid}).
To suppress backgrounds from events with more than two leptons, events with more than two leptons passing the preselection are vetoed.

Missing transverse energy ($\met$) represents any particles that escape the detector without being measured, such as neutrinos, and is defined as the magnitude of the vector sum of transverse momenta of all reconstructed objects.
It can be difficult to calculate accurately, as it involves measurements from all subsystems within the detector, and it is sensitive to any corrections that may be applied to the reconstructed physics objects~\cite{2018.met-13tev}.
These corrections, including the momentum smearing for muons, energy scale and smearing for electrons, and jet calibrations, are propagated to the $\met$ calculation.
Events are required to contain $\met > 30\gev$ in order to account for the two neutrinos from the $W$ boson decays.

At least two jets are required.
The leading and subleading jets must have $\pt > 65\gev$ and $\pt > 35\gev$, respectively, and are referred to as the \emph{tagging jets}.
Events are vetoed if they contain one or more jets that have been tagged as a $b$-jet to suppress backgrounds from heavy flavor decays (especially top quark events).
The $b$-tagging algorithm used by ATLAS is a boosted decision tree (BDT) called MV2c10, and this analysis uses a workingpoint with 85\% efficiency~\cite{2018.btag-efficiency-13tev}.

Finally, cuts are applied on the VBS signature outlined in Section~\ref{ssww13tev:ssww_topology}.
The tagging jets are required to have a dijet invariant mass $m_{jj} > 200\gev$ and be separated in rapidity by $|\Delta y_{jj}| > 2.0$.
This preferentially selects the VBS EWK events over the QCD-produced \ssww events.

\begin{table}[htbp]
  \centering
  \begin{tabular}{l | l}
    \multicolumn{2}{c}{Event selection} \\
    \hline\hline
    \multirow{3}{*}{Event preselection} & Pass at least one trigger with a matched lepton \\
                                        & Pass event cleaning \\
                                        & At least one reconstructed vertex\\
    \hline
    \multirow{4}{*}{Lepton selection}   & Exactly two leptons passing signal selection \\
                                        & Both signal leptons with the same electric charge \\
                                        & $|\eta| < 1.37$ and $|M_{ee} - M_Z| > 15\gev$ ($ee$-channel only) \\
                                        & Veto events with more than two preselected leptons \\
    \hline
    Missing transverse energy           & $\met \ge 30\gev$\\
    \hline
    \multirow{6}{*}{Jet selection}      & At least two jets \\
                                        & Leading jet $\pt > 65\gev$ \\
                                        & Subleading jet $\pt > 35\gev$\\
                                        & $m_{jj} > 200\gev$ \\
                                        & $N_{b\textrm{-jet}} = 0$ \\ %$b$-jet veto \\
                                        & $|\Delta y_{jj}| > 2.0$ \\
    \hline
  \end{tabular}
  \caption{The signal event selection.}
  \label{tab:ssww13tev_event_selection}
\end{table}
