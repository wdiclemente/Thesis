%\subsection{Data samples}\label{ssww13tev:data}
This analysis uses $36.1~\textrm{fb}^{-1}$ of \com{13} proton-proton collision data recorded by ATLAS during 2015 and 2016.
The uncertainty in the combined 2015+2016 integrated luminosity is 2.1\%. It is derived following a methodology similar to that detailed in Ref.~\cite{2016.atlas-luminosity-8tev} and using the LUCID-2 detector for the baseline luminosity measurements \cite{2018.atlas-luminosity-lucid2} from calibration of the luminosity scale using $x$-$y$ beam-separation scans. % the TWIKI literally says I can copy these lines verbatim for papers/conf notes :D https://twiki.cern.ch/twiki/bin/view/Atlas/LuminosityForPhysics#2017_13_TeV_proton_proton_Morion

\subsection{Monte Carlo samples}\label{ssww13tev:mc}
A number of Monte Carlo (MC) simululations are employed to model signal and background processes.
In order to model the real collision data as closely as possible, each MC has been run through a full simulation of the ATLAS detector~\cite{2010.ATLAS-simulation-infrastructure} in \tt{GEANT4}~\cite{2003.GEANT4}, and events have been reconstructed using the same algorithms as the data.
The simulation reproduces as closely as possible the momentum resolutions and calorimeter responses of the detector, and also includes the effects of pileup by including soft QCD interactions using \pythiav{8.1}~\cite{2008.Pythia8}.
The MC samples used in this analysis are detailed in this section and summarized in Table~\ref{tab:ssww13tev_mcsamples}.

The \ssww samples are modeled using \sherpav{2.2.2}~\cite{2009.Sherpa, 2008.CS_Shower, 2009.METS} with the \nnpdf PDF set~\cite{2015.NNPDF3}.
The EWK signal samples are generated by fixing the electroweak coupling constant to $\mathcal{O}(\alpha_W) = 6$, and a QCD background sample was also generated with $\mathcal{O}(\alpha_W) = 4$.
\tt{SHERPA} includes up to one parton at next-to-leading order (NLO) and up to three at leading order (LO) in the strong coupling constant $\alpha_s$.
A second \ssww EWK sample is generated using \powhegbox{2}~\cite{2010.powhegbox} with the \nnpdf PDF set and at NLO accuracy.
This sample is only used for systematic studies, as \tt{POWHEG-BOX} does not include resonant triboson contributions in its matrix element, which are non-negligible at NLO~\cite{2018.ssww-scattering-at-lhc}.

Diboson processes ($VV$ where $V = W,Z$) are simulated with \sherpav{2.2.2} for mixed hadronic and leptonic decays and \sherpav{2.2.1} for fully leptonic decays of the bosons.
Similarly, triboson ($VVV$) and $V\gamma$ processes are simulated using \sherpav{2.1.1} with up to one parton at NLO and up to three at LO.
$W$+jets processes are simulated with \sherpa{2.2.1} with up to two partons at NLO and four at LO.
All the above \tt{SHERPA} samples use the \nnpdf PDF set and \tt{SHERPA}'s own parton showering.
The $Z$+jets events are generated with \mcatnlo~\cite{2014.madgraph_mcnlo} at LO and interfaced with \pythiav{8.1} for parton showering.

$t\bar{t}$ events are generated using \powhegbox{2} with the \ctten PDF set~\cite{2010.ct10}.
$t\bar{t}V$ samples are generated at NLO with \mcatnlo and the \nnpdf PDF set interfaced with \pythiav{8} for parton showering.
Finally, single top events are generated with \powhegbox{1} and the \tt{CT10f4} PDF set interfaced with \pythia{6}~\cite{2006.Pythia6} for parton showering.

\begin{table}
  \centering
  \begin{tabular}{l l l}
    Process & Generator & Comments\\
    \hline\hline
    \ssww (EWK) & \sherpav{2.2.2} & Signal sample \\
    \ssww (EWK) & \powhegbox{2}   & Systematics sample \\
    \ssww (QCD) & \sherpav{2.2.2} & \\
    \hline
    \multirow{2}{*}{Diboson} & \sherpav{2.2.2} & Both bosons decay leptonically ($llll$, $lll\nu$, $ll\nu\nu$)\\
                             & \sherpav{2.2.1} & One boson decays leptonically, the other hadronically\\
    Triboson                 & \sherpav{2.1.1} & \\
    \hline
    $W$+jets           & \sherpav{2.2.1} & \\
    $Z$+jets           & \mcatnlo        & \\
    $V\gamma$          & \sherpav{2.1.1} & \\
    $V\gamma jj$ (EWK) & \sherpav{2.2.4} & \\
    \hline
    $t\bar{t}V$ & \mcatnlo        & \\
    $t\bar{t}$  & \powhegbox{2}   & \\
    Single top  & \powhegbox{1}   & EWK $t$-, $s$-, \& $Wt$-channels\\
    \hline
  \end{tabular}
  \caption{Summary of MC samples used in the analysis.}
  \label{tab:ssww13tev_mcsamples}
\end{table}
