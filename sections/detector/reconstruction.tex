Particle reconstruction algorithms

\subsubsection{Track reconstruction}\label{detector:track_reconstruction}
The track reconstruction algorithm used in ATLAS follows 

Following this procedure, TRT hits can be incorporated into the track fit through \emph{TRT track extension}~\cite{2008.newt}.
Compatible sets of TRT measurements are found for tracks found in the silicon detectors surviving the ambiguity solving.
The algorithm requires that the original silicon-only track not be modified by the inclusion of the TRT hits; it is simply an extension of the existing track.

What is described above is the \emph{inside-out} reconstruction algorithm; there is also an \emph{outside-in} reconstruction that begins in the TRT.
This algorithm is not covered in detail here, as much of the process is similar to the above.
The general workflow begins with finding track segments in the TRT, constructing the track candidates including the silicon hits, and finally ambiguity solving.

\subsubsection{Muon reconstruction}\label{detector:muon_reconstruction}

\subsubsection{Electron reconstruction}\label{detector:electron_reconstruction}

\subsubsection{Jet reconstruction}\label{detector:jet_reconstruction}
