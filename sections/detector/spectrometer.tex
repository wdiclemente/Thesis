The outermost subdetector in ATLAS is the Muon Spectrometer~\cite{1997.ms-tdr}.
Due to the high mass of muons compared to electrons, they pass through the calorimeters, necessitating their own detector.
The MS is a high-resolution spectrometer which provides tracking for muon reconstruction within $|\eta| < 2.7$.
A set of toroid magnets generate an azimuthal magnetic field that bends the muons for momentum measurements, much like in the ID.
Four different technologies are used in the MS:
\begin{itemize}
\item Monitored Drift Tubes (MDT) are used across the entire $\eta$ range for precision measurements of the tracks with a per-hit resolution in the range of 60-80~$\mum$.
These consist of an aluminum tube filled with a gas mixture containing argon and an anode wire running through the middle of the tube.
When a muon passes through, the gas is ionized, and the electrons are collected on the wire.
\item Cathode Strip Chambers (CSC) are used for the forward regions of the endcaps (above $|\eta| > 2.0$).
They operate on a similar principle to the MDTs, with strips containing a mesh of anode wires running in parallel instead of tubes with a single wire each.
\item Resistive Plate Chambers (RPC) in the barrel are primarily used to provide input for the muon trigger system.
They consist of pairs of plastic resistive plates with a 2~mm gap between them filled with a gas mixture.
Electrodes are attached to the plates to create a potential between them, and muons passing through ionize the gas and lead to electric discharges which in turn reduce the potential.
\item Thin Gap Chambers (TGC) are used for triggering in the endcaps.
The TGCs are arranged on circular disks consisting of two rings, and are similar in function to the CSCs but with a different gas mixture.
\end{itemize}
