ATLAS (A Toroidal L{\scshape hc} ApparatuS) is a general-purpose particle detector.
It is made up of several different subdetector systems designed to measure different types of particles.
Starting from the beam line and working outwards, the Pixel Detector (PIX), Semiconductor Tracker (SCT), and Transition Radiation Tracker (TRT) make up the Inner Detector (ID) and are responsible for measuring the trajectories and momenta of charged particles.
Next are two calorimeters: the Liquid Argon Calorimeter (LAr) stops electromagnetic objects and measures their energies, and the Tile Calorimeter does the same for hadronic objects.
Finally, the outermost Muon Spectrometer (MS) measures muon tracks as they leave the detector, as they are not stopped in the calorimeters.

ATLAS uses a global, right-handed coordinate system with the origin at the center of the detector (the nominal interaction point).
The $x$-axis points from the origin inwards to the center of the LHC ring, the $y$-axis points upwards, and the $z$-axis points along the beam line.
Due to the azimuthal symmetry of the detector, it is useful to use cylindrical coordinates ($r$, $\phi$) in the plane transverse to the $z$-axis where $\phi$ is the azimuthal angle.
Instead of using the polar angle $\theta$ to describe particle trajectories, pseudorapidity $\eta$ is used instead, defined in terms of $\theta$:
\begin{equation}
  \eta = -\ln(\tan(\theta/2))
  \label{eq:eta}
\end{equation}
Pseudorapidity has the useful property that differences in $\eta$ are invariant under Lorentz boosts along the $z$-axis.
The separation between two particles $p_i$ and $p_j$ is often expressed in terms of the quantity $\Delta R$, defined as:
\begin{equation}
  \Delta R(p_i,p_j) = \sqrt{(\eta_i-\eta_j)^2 + \min\big[|\phi_i-\phi_j|, |\phi_j-\phi_i|\big]^2}
  \label{eq:deltar}
\end{equation}
Here, the difference in $\phi$ is taken in whichever direction results in a value less than $\pi$, since $\phi = [0,2\pi]$ ``wraps around'' the detector in the azimuthal direction.

\begin{figure}[tbp]
  \begin{center}
    \includegraphics[width=0.98\textwidth]{figs/detector/atlas.pdf}
  \end{center}
  \caption[Cut-away view of the ATLAS detector.]{Cut-away view of the ATLAS detector~\cite{PERF-2007-01}.}
\end{figure}
