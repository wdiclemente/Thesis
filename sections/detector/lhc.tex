The Large Hadron Collider (LHC) \cite{2008.LHC} is the most powerful particle accelerator in the world in terms of beam energy, colliding two beams of protons at a center of mass energy of \com{13}.
It is operated by the European Organization for Nuclear Research (CERN), and the collider is located beneath the France--Switzerland border.
The LHC itself consists of a 27~km ring in which the collisions occur, and it is the last piece in a chain of several smaller accelerators that begin boosting the protons\footnote{The LHC can also collide beams of heavy ions; however, this thesis focuses exclusively on the proton-proton collisions.} to high energies.
Collisions occur at each of four detector experiments situated around the ring: ATLAS~\cite{2008.ATLAS.detector-paper}, ALICE~\cite{2008.alice}, CMS~\cite{2008.cms}, and LHCb~\cite{2008.lhcb}.

Protons are obtained from hydrogen atoms stripped of their electrons by an electric field.
A beam of protons is first accelerated up to $50\mev$ in the Linac 2 accelerator, then to $1.4\gev$ in the Proton Synchroton Booster (PSB), $25\gev$ in the Proton Synchrotron (PS), and finally to $450\gev$ in the Super Proton Synchrotron (SPS).
The protons are now injected into the LHC ring in two beams running in opposite directions where they each accelerate up to the collision energy of $6.5\tev$.
The beams consist of bunches containing on the order of $10^{11}$ protons separated by $25~\textrm{ns}$~\cite{2019.accelerator-complex}.
A schematic of the CERN accelerator complex, including the chain of accelerators mentioned above, is shown in Figure~\ref{fig:detector_accelerator_complex}.

\begin{figure}
  \centering
  \includegraphics[width=.9\textwidth]{figs/detector/accelerator-complex-small}
  \caption[The CERN accelerator complex.  For LHC collisions, protons are accelerated in the PSB (purple), the PS (magenta), and the SPS (light blue) before entering the LHC ring (dark blue).]{The CERN accelerator complex.  For LHC collisions, protons are accelerated in the PSB (purple), the PS (magenta), and the SPS (light blue) before entering the LHC ring (dark blue)~\cite{2016.accelerator-image}.}
  \label{fig:detector_accelerator_complex}
\end{figure}

In addition to a high center of mass energy, the LHC must also deliver enough data to measure rare processes.
The amount of data collected is measured in terms of \emph{luminosity}.
The instantaneous luminosity $\mathcal{L}$ is defined in terms of the number of events per section $\deriv{R}{t}$ and the production cross section $\sigma_p$:
\begin{equation}
  \mathcal{L} = \frac{1}{\sigma_p}\deriv{R}{t}
  \label{eq:inst_lumi}
\end{equation}
The calculation itself can be quite tricky, as it depends on a number of factors including (but not limited to) the number of particles per bunch, the spread of the beam, and the crossing angle of the beam~\cite{2006.lumi}.

The LHC was originally designed to operate at an instantaneous luminosity of $1.0\times 10^{34}~\textrm{cm}^{-2}\textrm{s}^{-1}$; however, this number was exceeded by the end of the 2016 data taking period, with a peak luminosity of $1.38\times 10^{34}~\textrm{cm}^{-2}\textrm{s}^{-1}$.
This number has been more than doubled by the end of Run 2 in December of 2018~\cite{2019.atlas-lumi-plots}.
The instantaneous luminosity of $pp$ collisions as a function of time in 2015 and 2016 are shown in Figure~\ref{fig:detector_instantaneous_lumi}.
The integrated luminosity is then the time integral of the instantaneous luminosity.
By the end of Run 2 (2015-2018), approximately $140~\textrm{fb}^{-1}$ of $13\tev$ data is available for physics, as shown in Figure~\ref{fig:atlas_integrated_lumi}.
The $36.1~\textrm{fb}^{-1}$ collected during the first two years (2015 and 2016) is used for the analysis later in this thesis.

\begin{figure}[htbp]
  \centering
  \begin{subfigure}[b]{.48\textwidth}
    \includegraphics[width=\textwidth]{figs/detector/peakLumiByFill2015}
    \caption{2015}
  \end{subfigure}
  \begin{subfigure}[b]{.48\textwidth}
    \includegraphics[width=\textwidth]{figs/detector/peakLumiByFill2016}
    \caption{2016}
  \end{subfigure}
  \caption[Peak instantaneous luminosity delivered to ATLAS during $13\tev$ $pp$ data taking as a function of time.]{Peak instantaneous luminosity delivered to ATLAS during $13\tev$ $pp$ data taking as a function of time~\cite{2019.atlas-lumi-plots}.}
  \label{fig:detector_instantaneous_lumi}
\end{figure}

\begin{figure}
  \centering
  \includegraphics[width=.6\textwidth]{figs/detector/intLumiVsTimeRun2}
  \caption[Integrated luminosity collected by ATLAS as a function of time at $13\tev$ from 2015-2018.]{Integrated luminosity collected by ATLAS as a function of time at $13\tev$ from 2015-2018~\cite{2019.atlas-lumi-plots}.}
  \label{fig:atlas_integrated_lumi}
\end{figure}

Due to the high instantaneous luminosity, more than one $pp$ interaction occurs in a single bunch crossing, referred to as \emph{pileup}.
During the 2016 data taking campaign, the average number of interactions per bunch crossing $<\mu >$ was approximately 24 but has increased to upwards of 37 in 2017 and 2018~\cite{2019.atlas-lumi-plots}.
Figure~\ref{fig:detector_pileup} contains the average $\mu$ for the 2015-2016 data set used for analysis in this thesis.
The high pileup is a challenge for accurately reconstructing an individual collision.

\begin{figure}
  \centering
  \includegraphics[width=.6\textwidth]{figs/detector/meanInteractionsPerCrossing}
  \caption[Distribution of the mean number of interactions per bunch crossing for the 2015 and 2016 $pp$ collision data at $13\tev$.]{Distribution of the mean number of interactions per bunch crossing for the 2015 and 2016 $pp$ collision data at $13\tev$~\cite{2019.atlas-lumi-plots}.}
  \label{fig:detector_pileup}
\end{figure}
